\documentclass[a4paper,12pt,english]{scrartcl}

% Language
\usepackage{polyglossia}
\setmainlanguage[variant=american]{english}

\usepackage[math-style=TeX]{unicode-math}
\usepackage{graphicx}

\usepackage{isodate}
\origdate % Make today print in another format

% Geiler header
\usepackage{fancyhdr}
\fancyhf{} % clear all header and footer fields
\fancyhead[L]{{\small \textsf{[WS 14/15, NYT] Homework 1}}}
\fancyhead[R]{{\small \textsf{12th October 2014}}}

% Text in englischen Anführungszeichen mit \enquote{Very important}
\usepackage[autostyle=true,english=american]{csquotes}

% Ränder auf 3cm setzen
\usepackage[paper=a4paper,left=30mm,right=30mm,top=30mm,bottom=30mm]{geometry}

\begin{document}
\pagestyle{fancy} % eigenen Seitestil aktivieren

\noindent
\textbf{\textsf{{\large [F74] F Feynman, Richard P. (1974). \enquote{Cargo Cult Science}. Engineering and Science, 1974.}}}
\vspace{2mm}

The author begins by providing a short history how science evolved: by
debunking myths. Science as a method to test if certain isolated ideas work or
not. He decided to investigate why people believed in witch doctors and alike
and found out, that non of those phenomena actually work.

The question \enquote{what else is there what we believe?} arises: how to
treat criminals, how we educate and various other pseudo-sciences. The author
introduces the term \enquote{Cargo Cult Science} describing sciences that are
actually not scientific and practices that don't work.

Futhermore, the question how to be scientifc is answered. Provide all the
information to judge or value a contribution, not just the information
that you would like to provide and concludes that we don't actually teach this
scientific integrity in class.

He advices that we should not fool the layman or one seeking advice from us.
Always be honest and conduct experiments unbiased. This involves making all
results public, not just those we want to in order to achieve something such
as reputation or alike.
Sometimes, this scientific integrity or utter honesty is disallowed by
political issues or other challenges in life and sometimes even lazyness.

\vspace{8mm}

\noindent
\textbf{\textsf{{\large [H+04] Denning, Peter J. (2005). \enquote{Is Computer Science a Science?} CACM vol. 48, no. 4 (April 2005), pp 27-31.}}}
\vspace{2mm}

In his paper \enquote{Is Computer Science Science?}, Peter J. Denning tries
to answer the question wether computer science is an actual scientific field.
The rationale behind the question is easily explained: all sciences are about
fundamental laws of nature whereas the subject of computer science is
man-made.

The author explains that computer science is a blend of three disciplines:
computing science, engineering and mathematics. As any other science, it
proposes hypotheses and test them through experiments. However, computing
science does not study man-made objects. It studies information processes.
The computers \textemdash{} as in many other fields such as physics or chemistry
\textemdash{} are just the tools for conducting efficient research.

Depending on which research area in computing people come from, they have
different opinions weather computer science is a science or an art. The author
concludes that there it is both. He then questions if computer science has
elements other scientific areas have: depth, a future and credibility.

Depth meaning computer science has fundamental principles: it has; a future:
yes, since it combines with other disciplines to create new research areas.
The problem of credibility however is self-inflicted. Too many times the field
of computer science has claimed to be able to do something but ultimately
failed to deliver (i.e. artificial intelligence). This kind of marketing
however does not adhere to the scientific standard. Fortunately, this problem
is fading away, as younger scientists enter the field.
\end{document}
