\documentclass[a4paper,12pt,english]{scrartcl}

% Language
\usepackage{polyglossia}
\setmainlanguage[variant=american]{english}

\usepackage[math-style=TeX]{unicode-math}
\usepackage{graphicx}

% \todo{Add this.}
\usepackage[backgroundcolor=red]{todonotes}

% Text in englischen Anführungszeichen mit \enquote{Very important}
\usepackage[autostyle=true,english=american]{csquotes}

% Ränder auf 3cm setzen
\usepackage[paper=a4paper,left=30mm,right=30mm,top=30mm,bottom=30mm]{geometry}

% Titelfont aus normal setzen
\setkomafont{disposition}{\normalfont}

% Titel höher
\usepackage{titling}
\setlength{\droptitle}{-5em}

% Zeilenabstand 1.5
\linespread{1.5}

\title{Is Computer Science Science? \\ {\large A summary}}

\begin{document}
\maketitle

% Keine Seitenzahlen
\thispagestyle{empty}
\pagestyle{empty}

In his paper \enquote{Is Computer Science Science?}, Peter J. Denning tries
to answer the question wether computer science is an actual scientific field.
The rationale behind the question is easily explained: all sciences are about
fundamental laws of nature whereas the subject of computer science is
man-made.

The author explains that computer science is a blend of three disciplines:
computing science, engineering and mathematics. As any other science, it
proposes hypotheses and test them through experiments. However, computing
science does not study man-made objects. It studies information processes.
The computers \textemdash{} as in many other fields such as physics or chemistry
\textemdash{} are just the tools for conducting efficient research.

Depending on which research area in computing people come from, they have
different opinions weather computer science is a science or an art. The author
concludes that there it is both. He then questions if computer science has
elements other scientific areas have: depth, a future and credibility.

Depth meaning computer science has fundamental principles: it has; a future:
yes, since it combines with other disciplines to create new research areas.
The problem of credibility however is self-inflicted. Too many times the field
of computer science has claimed to be able to do something but ultimately
failed to deliver (i.e. artificial intelligence). This kind of marketing
however does not adhere to the scientific standard. Fortunately, this problem
is fading away, as younger scientists enter the field.

\end{document}
