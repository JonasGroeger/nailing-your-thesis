\documentclass[a4paper,12pt,english]{scrartcl}

% Language
\usepackage{polyglossia}
\setmainlanguage[variant=american]{english}

% Make \today print in format "NN<st|nd|rd|th> MM YYYY"
\usepackage{isodate}
\origdate

\linespread{1.15}

\usepackage{titlesec}
\titleformat{\section}{\large\bfseries\sffamily}{}{0pt}{}

% Write something in bold, sans-serif and a little larger to indicate the title
% Usage: \papertitle{DP79}{Title of the paper}
\newcommand{\papertitle}[2]{
	\section{[#1] #2}
}

% Own pagestyle (header and footer)
\usepackage{fancyhdr}
\fancyhf{} % Clear header and footer content
\fancyhead[L]{{\small \textsf{WS 14/15, NYT}}}
\fancyhead[C]{{\small \textsf{Homework 9}}}
\fancyhead[R]{{\small \textsf{\today}}}
\fancyfoot[C]{\thepage}

% Text in quotes
% Usage: \enquote{To be or not to be.}
\usepackage[autostyle=true,english=american]{csquotes}

% Subliminal refinements towards typographical perfection
\usepackage{microtype}

\usepackage{blindtext}

% Borders
\usepackage[paper=a4paper,left=20mm,right=20mm,top=30mm,bottom=30mm]{geometry}
\setlength{\parskip}{0.4em}

\begin{document}
\pagestyle{fancy} % Activate own pagestyle

\papertitle{B+10}{Beel, J., Gipp, B., \& Wilde, E. (2010). Academic search engine optimization (ASEO). Journal of scholarly publishing, 41(2), 176-190.}

In general, authors of papers want their papers to be found. It will get found if it is indexed and it has a high ranking.
Therefore, Academic Search Engine Optimization was invented to help one in having a paper indexed and ranked high.

In comparison to the traditional SEO of websites, ASEO is more difficult due to some specifics related to the practices of scientific publishing:
academic search engines work differently and publishing is almost exclusively done in PDF.

Different academic search engines use different algorithms to calculate the relevance of a paper.
The higher the relevance, the higher a paper will be ranked.
That leads to a higher placement in the search results that a user gets displayed.
The academic search engine Google Scholar puts high relevance on papers whose title, author or publication match the user query.
The citation count of a paper is another indicator how high the relevance will be.
Aside from the text, images can only be considered if they are in a vector-based format.
Google Scholar gathers the metadata of papers using an invite-based system where authors of affiliates can register their paper.
The search engine then finds matching PDFs on the internet, which might also include older versions of a paper.

Having understood the algorithm on a broad level, one can start optimizing his publications to have higher relevance.
That means
(1) picking non-popular keywords,
(2) putting the keywords in important places like the title, abstract or the PDF metadata,
(3) vectorizing images and
(4) publishing the paper to open-access or to affiliates of Google Scholar.

The paper has been criticized of making authors focus more on ranking rather than on scientific impact.
However, this problem is one that traditional search engines also had when SEO got popular and will be solved by better filtering algorithms in ASEOs.

\newpage

\papertitle{S03}{Shaw, M. (2003, May). Writing good software engineering research papers: minitutorial. In Proceedings of the 25th international conference on software engineering (pp. 726-736). IEEE Computer Society.}

This paper wants to introduce the reader on how to write a good paper in the field of software engineering.
In general, papers are contributions to the scientific community and show scientific progress.
Writing a paper, one should be clear and precise about three main elements:

\begin{enumerate}
	\item \textbf{Research question:} What is the problem you want to solve? Why does that matter?
	\item \textbf{Result:} What have you added to the body of software engineering knowledge?
	\item \textbf{Validation:} Proof that it works in a more general context; outside of your environment.
\end{enumerate}

First, you should make your research question clear.
State the problem that you are trying to solve.
The paper explains the most common types of research questions and analyzes their accept / reject rate at a ICSE 2002, a software engineering conference.

Second, explain what you have added to the body of knowledge.
Again, the paper explains different types of research results and analyzes their accept / reject rate at ICSE 2002.
Committees want your claims to be reasoned appropriately.
Also, the ralation of your work in relation to the work of others should be explained.

Third, a scientific contribution has to create trust among researchers.
This paper explains various means of validation and again, measures their accept / reject rate at the same conference.
In general, committees look for solid evidence that supports the claims you made in your result.

The process of building and managing 1., 2. and 3. is called a research strategy.
However, not all tuples of \enquote{type of research question}, \enquote{type of research result} and \enquote{type of validation} are appropriate.

The last important point of this paper is that abstracts are important and the body of a paper should deliver what its abstract promises.

\end{document}
