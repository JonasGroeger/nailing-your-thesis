\documentclass[a4paper,12pt,english]{scrartcl}

% Language
\usepackage{polyglossia}
\setmainlanguage[variant=american]{english}

% Make \today print in format "NN<st|nd|rd|th> MM YYYY"
\usepackage{isodate}
\origdate

\linespread{1.15}

\usepackage{titlesec}
\titleformat{\section}{\large\bfseries\sffamily}{}{0pt}{}

% Write something in bold, sans-serif and a little larger to indicate the title
% Usage: \papertitle{DP79}{Title of the paper}
\newcommand{\papertitle}[2]{
	\section{[#1] #2}
}

% Own pagestyle (header and footer)
\usepackage{fancyhdr}
\fancyhf{} % Clear header and footer content
\fancyhead[L]{{\small \textsf{WS 14/15, NYT}}}
\fancyhead[C]{{\small \textsf{Homework 5}}}
\fancyhead[R]{{\small \textsf{\today}}}
\fancyfoot[C]{\thepage}

% Text in quotes
% Usage: \enquote{To be or not to be.}
\usepackage[autostyle=true,english=american]{csquotes}

% Subliminal refinements towards typographical perfection
\usepackage{microtype}

% Borders
\usepackage[paper=a4paper,left=20mm,right=20mm,top=30mm,bottom=30mm]{geometry}
\setlength{\parskip}{0.4em}

\usepackage{hyperref}

\begin{document}
\pagestyle{fancy} % Activate own pagestyle

\papertitle{E89}{Eisenhardt, K. M. (1989). Building theories from case study research. Academy of management review, 14(4), 532-550.}

% Introduction

Case study research tries to understand a phenomenon by looking at it \enquote{in the wild}, represented as one or more case studies. This paper introduces a process for building theories from case study research. Furthermore, it contributes to position case study research into social science research.

According to this paper, the process of building theory from case study research can be divided in 8 steps. Getting started (1) is about the definition of a research question to not be overwhelmed by the data. Even if it may not be the final research question, it helps in creating a research focus. Selecting cases (2) describes the issues in selecting the case studies to base the emergant theory upon. The general idea is to choose extreme cases that will make the theory-in-process more visible. Crafting instruments and protocols (3) argues that theory-building researchers should use multiple data sources to achieve good quality and to remove biases. Entering the field (4) discusses the idea that data collection and analysis must not happen in sequence but rather in a blended fashion. This can provide the researcher with information wether he should make adjustments to his data collection methods. Analyzing the data (5) is a rather fuzzy step. In order to cope with the massive amount of data, write-ups of each element under investigation (the paper calls them \enquote{sites}) should be created in order to better understand them. Analyzing, one should also look for pieces of evidence that appeared in one site but did not in another. In \enquote{Shaping hypotheses} (6), a researcher should iteratively merge theory and data in order to come up with a theory that closely fits the data. Constructs supported by multiple data items should be created as they are the building blocks for the theory. Cross-construct relationships should also be checked for validity. Enfolding literature (7) issues some guidelines on what do do with existing literature regarding the same or similar theories or hypothesis. There, finding contradictions and similarities can be helpful. They should be addressed as they improve the quality and confidence in the theory-in-process. Reaching closure (8) handles the topic that a theory can never be perfect and fit all cases. The general idea here is that there will be a saturation point where new data only improves the theory so little that it is no longer worthwhile.

Last, the paper discusses strength and weaknesses of theory building from cases. A central strength of that method (among others) is that it is very likely to create new theories. In contrast, the theories created are often overly complex and possibly narrow regarding applicability.

The paper concludes, stating that there is \enquote{no generally accepted set of guidelines for the assessment of case study research}. However, good case study research creates novel, testable, coherent and sometimes groundbreaking theory.
\newpage

\papertitle{EG07}{Eisenhardt, K. M., \& Graebner, M. E. (2007). Theory building from cases: opportunities and challenges. Academy of management journal, 50(1), 25-32.}

This paper is about case study research. Specifically, it talks about the challenges that a researcher has to overcome when building theory from case studies. It also adresses the opportunities that arise from that kind of research.

The first challenge is concerned with the justification for doing theory building rather than theory testing. A common strategy to apply here is to question the one in doubt why there is not already a theory covering most or all phenomena. Most of the time the lack of theories is the exact reason for doing theory building rather than testing existing ones.

Case selection is another challange the paper discusses. In general, case study research is used to develop theories. Therefore it is required to have some case studies to work with. The more case studies a theory is based upon, the stronger the theory of that phenomenon will be. Then however, the theory will also be less deep and will not cover all details of all case studies. If one tries to include all edge cases of the case studies, the theory will just get more complicated. The opposite effect occurs when using only a single case study to base a theory on.

Interview data is one of the most common techniques of collecting data in case studies. The quality of gathered data is very high, however, the data will be biased. It is recommended to use multiple knowledgeable persons with a good understanding of the overall phenomenon circumstances. Cross-checking multiple interviews will most probably remove a large chunk of biases. To complement the interviews, doing observations is adviced. They too will provide the researcher with more data as well as remove some biases from interviews (people think differently when outside of an interview situation).

The presentation of the empirical evidence is the next challenge. Depending on the number of case studies used, the stories in them cannot be added to the publication. To address this issue, it is adviced to try to blend story and theory in a so called \enquote{construct table} containing theory constructs and text from multiple cases to each construct.

Then an opportunity is presented regarding the composal of the theory. In the publication, one should present it in three different ways: (1) a sketch of the theory in the introduction, (2) each construct of the theory in the body with empirical evidence, (3) use \enquote{boxes and arrows} to present the theory visually.

Despite these challenges, case study research remains popular and important. The challenges presented can be met through sound design and precise language.

\end{document}


%%%%%%%%%%%%%%%%%%%%%%%%%%%%%%%%%%%%%%%%%%%%%%%%%%%%%%%%%%%%%%%%%%%%%%%%%%%%%%%
% Case studies
	% Definition
	% Building theory from them
% Challenges
	% Quality (Too much narration / too much labeling) depends on the researcher and where he is coming from
	% Justification for doing theory building rather than theory-testing
		% Ask, why there is no existing theory that offers feasible answers
	% Case selection
		% Case study research are used to develop theory, not to test it.
		% Theory of a phenomenon gets stronger (but also less detailed and more complicated) if more cases are used.
	% Interview data is good but complicated (hard to remove biases)
		% Use multiple knwledgable informants (CEOs, board members) with a good understanding of the overall phenomenon circumstances.
		% Combination of interview & observation is good
	% Presenting Empirical Evidence
			% In a single-case study we can easily present qualitative data in form of a story.
			% In a multiple-case study this is not possible but creating stronger theory is.
			% The paper suggests that one should try to blend story and theory in "construct tables". (Theory constructs with text from multiple cases)
	% Writing the emergent theory
		% Write the theory in three ways: Sketch in introduction, each construct in the body with empirical evidence. Then use a drawing using "boxes and arrows".

% Conclusion
	% Theory building from case studies is popular and relevant
	% Some predictable challenges can be met through sound design and precise language
%%%%%%%%%%%%%%%%%%%%%%%%%%%%%%%%%%%%%%%%%%%%%%%%%%%%%%%%%%%%%%%%%%%%%%%%%%%%%%%
