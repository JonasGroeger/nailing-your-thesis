\documentclass[a4paper,12pt,english]{scrartcl}

% Language
\usepackage{polyglossia}
\setmainlanguage[variant=american]{english}

% Make \today print in format "NN<st|nd|rd|th> MM YYYY"
\usepackage{isodate}
\origdate

\linespread{1.15}

\usepackage{titlesec}
\titleformat{\section}{\large\bfseries\sffamily}{}{0pt}{}

% Write something in bold, sans-serif and a little larger to indicate the title
% Usage: \papertitle{DP79}{Title of the paper}
\newcommand{\papertitle}[2]{
	\section{[#1] #2}
}

% Own pagestyle (header and footer)
\usepackage{fancyhdr}
\fancyhf{} % Clear header and footer content
\fancyhead[L]{{\small \textsf{WS 14/15, NYT}}}
\fancyhead[C]{{\small \textsf{Homework 4}}}
\fancyhead[R]{{\small \textsf{\today}}}
\fancyfoot[C]{\thepage}

% Text in quotes
% Usage: \enquote{To be or not to be.}
\usepackage[autostyle=true,english=american]{csquotes}

% Subliminal refinements towards typographical perfection
\usepackage{microtype}

% Borders
\usepackage[paper=a4paper,left=20mm,right=20mm,top=30mm,bottom=30mm]{geometry}
\setlength{\parskip}{0.8em}

\begin{document}
\pagestyle{fancy} % Activate own pagestyle

\papertitle{CS07}{J. Corbin and A. Strauss, \enquote{Introduction,} in Basics of Qualitative Research: Techniques and Procedures for Developing Grounded Theory, 3rd ed. Thousand Oaks, CA: Sage Publications, 2007, ch. 1.}

% Grounded theory
	% Chicago Interactionism
	% Pragmatism
	% Assumptions
	% Implications
% Why we do qualitative research
	% Feel for participants
	% Qual: fluid, evolving, dynamic
	% Quant: rigid, structured
% Researchers want to change the world for good

Grounded theory is a methodology to build theory from data derived from qualitytive analysis. The presented methodology's epistemology presented in this chapter is based on two different traditions: Chicago Interactionism and Pragmatism. Chicago Interactionism is the fact that human beings do not react to actions of others but interpret them and act based on this interpretation. Pragmatism Pragmatism rejects the idea that the function of thought is to describe or represent. It embraces the idea that thought is an instrument to predict, to take action or to solve problems. Because the world is complex and ambiguous we cannot create a big methodology that works for everyone and everything. Thus, there are assumptions behind our methodology (the authors list 16 assumptions about actions, interactions and others). These assumptions and the fact that events are always the result of multiple factors lead to methodological implications. The authors do not want to \enquote{reduce understanding of action/interaction [\ldots] to one explanation or [\ldots] scheme} but rather reveal basic concepts on different levels of abstractions. Then, one of the authors explains that he wants to develop knowledge that in turn will guide practice --- not just knowledge for the sake of having it.

In the next section it is explained why we do qualitative research. The reason: because the research question should dictate the method and because this kind of research allows researchers \enquote{to get at the inner experience of the participants}. This means that a participant is not just a number on a spreadsheet but a individual with a culture, a background etc. that all must be taken into consideration. Quantitative reseachers have rather rigid and structured methods whereas qualitative reseachers have more fluid, evolving and dynamic research methods and will even reject practices such as objectivity and include personal experience.

The authors conclude that people do research to change the world for good. The book should and cannot be perfect but will at least give you some insights on how to analyze data and how to do qualitative research. Unfortunately this book can only provide textual guidance. However, qualitative analysis can only be learned by doing it, much like a craft.

{\tiny I especially like the Note at the end of the first chapter about it being too complicated for a beginning text on qualitative research.}

\newpage

\papertitle{CS07}{J. Corbin and A. Strauss, \enquote{Practical Considerations,} in Basics of Qualitative Research: Techniques and Procedures for Developing Grounded Theory, 3rd ed. Thousand Oaks, CA: Sage Publications, 2007, ch. 2.}

The second chapter covers five topics: (1) choosing a research problem and formulating a research question, (2) sensivity when it comes to data analysis, (3) literature to gain background information and (4) theoretical frameworks for qualitative research.

In choosing a research problem, you first need to identify a problem area you would like to work in. The authors name four sources of research problems: (a) problems suggested or assigned by advisors, (b) problems derived from literature, (c) problems derived from experience and (d) problems that pop up from research itself. Formulating a research question is the next step. The question should neither be too broad (too many issues to look after), nor too narrow (not enough for background information / culture of the participants / etc.). At the same time, one has to know that you cannot cover all aspects of a problem, because the world is too complex (see chapter 1). That makes designing a research question hard.

In qualitative research, the researcher is able to use a lot of data sources. However, he has to make sure the quality of the material is appropriate. One method to gather data stated to great extend in the book are interviews. They have a high data density but at the same time it is difficult to remember every detail, depending on how unstructured the interview is. The second method described are observations which are even more complicated than interviews. The reason lies in human nature: we interpret what we see while the observed person has a different interpretation about what he or she is doing. Observations have such importance because people in interviews may say one thing but actually do something else in the real situation. The topic of sensitivity can be described as the opposite of objectivity. Every researchers brings their knwledge, biases etc. to research. This is what makes us sensitive to patterns in data, unusual measurements etc. which might be the desired research result.

Literature will increase our sensitivity, as we gain more experience and see how other researchers are doing their work. Also, literaure will feed ones mind to create new ideas, questions or alike.

The theoretical frameworks act as a justification to use particular methods (in contrast to quantitative research) or structure the research. Furthermore, it is important to note that technical literature can sometimes hinder research and stand between the researcher and the data.

As a conclusion I would like to say that qualitative (fluid, sensitive) and quantitative research (rigid, structured) are like the two sides of a coin. Both work on their own. The research process however, is quite the same: research problem, research question, data gathering and analysis, literature and frameworks.

\end{document}
