\documentclass[a4paper,12pt,english]{scrartcl}

% Language
\usepackage{polyglossia}
\setmainlanguage[variant=american]{english}

% Make \today print in format "NN<st|nd|rd|th> MM YYYY"
\usepackage{isodate}
\origdate

\linespread{1.15}

% Write something in bold, sans-serif and a little larger to indicate the title
% Usage: \papertitle{DP79}{Title of the paper}
\newcommand{\papertitle}[2]{
	\noindent
	\textbf{\textsf{{\large
		[#1] #2
	}}}
	\vspace{2mm}
}

% Own pagestyle (header and footer)
\usepackage{fancyhdr}
\fancyhf{} % Clear header and footer content
\fancyhead[L]{{\small \textsf{WS 14/15, NYT}}}
\fancyhead[C]{{\small \textsf{Homework 4}}}
\fancyhead[R]{{\small \textsf{\today}}}
\fancyfoot[C]{\thepage}

% Text in quotes
% Usage: \enquote{To be or not to be.}
\usepackage[autostyle=true,english=american]{csquotes}

% Borders
\usepackage[paper=a4paper,left=20mm,right=20mm,top=30mm,bottom=30mm]{geometry}

\begin{document}
\pagestyle{fancy} % Activate own pagestyle

\papertitle{CS07}{J. Corbin and A. Strauss, \enquote{Introduction,} in Basics of Qualitative Research: Techniques and Procedures for Developing Grounded Theory, 3rd ed. Thousand Oaks, CA: Sage Publications, 2007, ch. 1.}

% Grounded theory
	% Chicago Interactionism
	% Pragmatism
	% Assumptions

Grounded theory is a methodology to build theory from data derived from
qualitytive analysis. The presented methodology's epistemology presented in this
chapter is based on two different traditions: Chicago Interactionism and
Pragmatism. Chicago Interactionism is the fact that human beings do not react to
actions of others but interpret them and act based on this interpretation.
Pragmatism Pragmatism rejects the idea that the function of thought is to
describe or represent. It embraces the idea that thought is an instrument to
predict, to take action or to solve problems. Because the world is complex and
ambiguous we cannot create a big methodology that works for everyone and
everything. Thus, there are assumptions behind our methodology (the auhors list
16 assumptions about actions, interactions and others).

\vspace{8mm}

\papertitle{CS07}{J. Corbin and A. Strauss, \enquote{Practical Considerations,} in Basics of Qualitative Research: Techniques and Procedures for Developing Grounded Theory, 3rd ed. Thousand Oaks, CA: Sage Publications, 2007, ch. 2.}

Insert some text here!

\end{document}
