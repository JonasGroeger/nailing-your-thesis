\documentclass[a4paper,12pt,english]{scrartcl}

% Language
\usepackage{polyglossia}
\setmainlanguage[variant=american]{english}

\usepackage[math-style=TeX]{unicode-math}
\usepackage{graphicx}

% \todo{Add this.}
\usepackage[backgroundcolor=red]{todonotes}

% Text in englischen Anführungszeichen mit \enquote{Very important}
\usepackage[autostyle=true,english=american]{csquotes}

% Ränder auf 3cm setzen
\usepackage[paper=a4paper,left=30mm,right=30mm,top=30mm,bottom=30mm]{geometry}

% Titelfont aus normal setzen
\setkomafont{disposition}{\normalfont}

% Titel höher
\usepackage{titling}
\setlength{\droptitle}{-5em}

% Zeilenabstand 1.5
\linespread{1.5}

\title{Cargo Cult Science \\ {\large A summary}}

\begin{document}
\maketitle

% Keine Seitenzahlen
\thispagestyle{empty}
\pagestyle{empty}

The author begins by providing a short history how science evolved: by
debunking myths. Science as a method to test if certain isolated ideas work or
not. He decided to investigate why people believed in witch doctors and alike
and found out, that non of those phenomena actually work.

The question \enquote{what else is there what we believe?} arises: how to
treat criminals, how we educate and various other pseudo-sciences. The author
introduces the term \enquote{Cargo Cult Science} describing sciences that are
actually not scientific and practices that don't work.

Futhermore, the question how to be scientifc is answered. Provide all the
information to judge or value a contribution, not just the information
that you would like to provide and concludes that we don't actually teach this
scientific integrity in class.

He advices that we should not fool the layman or one seeking advice from us.
Always be honest and conduct experiments unbiased. This involves making all
results public, not just those we want to in order to achieve something such
as reputation or alike.
Sometimes, this scientific integrity or utter honesty is disallowed by
political issues or other challenges in life and sometimes even lazyness.

\end{document}
