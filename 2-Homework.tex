\documentclass[a4paper,12pt,english]{scrartcl}

% Language
\usepackage{polyglossia}
\setmainlanguage[variant=american]{english}

\usepackage[math-style=TeX]{unicode-math}
\usepackage{graphicx}

\usepackage{isodate}
\origdate % Make today print in another format

% Geiler header
\usepackage{fancyhdr}
\fancyhf{} % clear all header and footer fields
\fancyhead[L]{{\small \textsf{[WS 14/15, NYT] Homework 1}}}
\fancyhead[R]{{\small \textsf{12th October 2014}}}

% Text in englischen Anführungszeichen mit \enquote{Very important}
\usepackage[autostyle=true,english=american]{csquotes}

% Ränder auf 3cm setzen
\usepackage[paper=a4paper,left=30mm,right=30mm,top=30mm,bottom=30mm]{geometry}

\begin{document}
\pagestyle{fancy} % eigenen Seitestil aktivieren

\noindent
\textbf{\textsf{{\large [DP79] De Millo, Lipton, Perlis. 1979. Social processes and proofs of theorems and programs. Commun. ACM 22, 5 (May 1979), 271-280.}}}
\vspace{2mm}

The paper about the social processes that take place in mathematics regarding
proofs and how this process can or cannot be applied to verification of computer
programs. At first it is argued that computer programming should become more
like mathematics. One instance how that can be done is by automatic program
verification.

Computer programming should be more like mathematics.
	One instance how to do that is by automatic program verification.

Maths and proofs
	Maths is a social, informal, intuitive, organic human process, a community project.
	Maths uses proofs to verify a theorem.
	A proof can only probably express the truth.
	Those proofs and theorems have to be believed in.
	Belief is not generated by cold, mechanical formal logic.
	Belief is created by a social process that tries to make mathematicans feel condident about a theorem
	or a proof.

Social processes in mathematics
	A proof is a spoken message or at most a sketch.
	If the proof generates excitement, a polished version will be made.
	If the polished version gets published, more people read it.
	After a cooldown period, if a larger audience likes the proof, they may belive in it.
	The believers will paraphrase the proof and create multiple versions of the theorem.
	A believable theorem gets used as part of other proofs and in the real world.
	All these steps improve the confidence that the theorem or proof is correct.
	After enough battle-testing, the theorem is thought to be true in the classical sense of truth.

Proof or verification of programs
	Program verification will not play the same role in Computer Science as proofs in maths do.
	It does not have them because the verifications cannot really be read by a person.
	Thus, they cannot acquire credibility gradually (as in maths); either completely believe in them or dont.
	The analogy of program verification and mathematical proof fails.
	Program verification will fail because it does not have the social processes like maths.

\vspace{8mm}

\noindent
\textbf{\textsf{{\large [H+04] Hevner, March, Park, Ram. 2004. Design science in information systems research. MIS Q. 28, 1 (March 2004), 75-105.}}}
\vspace{2mm}

Text...

\end{document}


% Computer programming should be more like mathematics.
% 	One instance how to do that is by automatic program verification.

% Maths and proofs
% 	Maths is a social, informal, intuitive, organic human process, a community project.
% 	Maths uses proofs to verify a theorem.
% 	A proof can only probably express the truth.
% 	Those proofs and theorems have to be believed in.
% 	Belief is not generated by cold, mechanical formal logic.
% 	Belief is created by a social process that tries to make mathematicans feel condident about a theorem
% 	or a proof.

% Social processes in mathematics
% 	A proof is a spoken message or at most a sketch.
% 	If the proof generates excitement, a polished version will be made.
% 	If the polished version gets published, more people read it.
% 	After a cooldown period, if a larger audience likes the proof, they may belive in it.
% 	The believers will paraphrase the proof and create multiple versions of the theorem.
% 	A believable theorem gets used as part of other proofs and in the real world.
% 	All these steps improve the confidence that the theorem or proof is correct.
% 	After enough battle-testing, the theorem is thought to be true in the classical sense of truth.

% Proof or verification of programs
% 	Program verification will not play the same role in Computer Science as proofs in maths do.
% 	It does not have them because the verifications cannot really be read by a person.
% 	Thus, they cannot acquire credibility gradually (as in maths); either completely believe in them or dont.
% 	The analogy of program verification and mathematical proof fails.
% 	Program verification will fail because it does not have the social processes like maths.
