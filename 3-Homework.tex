\documentclass[a4paper,12pt,english]{scrartcl}

% Language
\usepackage{polyglossia}
\setmainlanguage[variant=american]{english}

% Make \today print in format "NN<st|nd|rd|th> MM YYYY"
\usepackage{isodate}
\origdate

\linespread{1.15}

% Write something in bold, sans-serif and a little larger to indicate the title
% Usage: \papertitle{DP79}{Title of the paper}
\newcommand{\papertitle}[2]{
    \noindent
    \textbf{\textsf{{\large
        [#1] #2
    }}}
    \vspace{2mm}
}

% Own pagestyle (header and footer)
\usepackage{fancyhdr}
\fancyhf{} % Clear header and footer content
\fancyhead[L]{{\small \textsf{WS 14/15, NYT}}}
\fancyhead[C]{{\small \textsf{Homework 3}}}
\fancyhead[R]{{\small \textsf{25th October 2014}}}
%\fancyfoot[C]{\thepage}

% Text in quotes
% Usage: \enquote{To be or not to be.}
\usepackage[autostyle=true,english=american]{csquotes}

% Borders
\usepackage[paper=a4paper,left=20mm,right=20mm,top=30mm,bottom=30mm]{geometry}

%%%%%%%%%%%%%%%%%%%%%%%%%%%%%%%%%%%%%%%%%%%%%%%%%%%%%%%%%%%%%%%%%%%%%%%%%%%%%%%%%%%%%%%%%%%%%%%%%%%%%%%%%%%%%%%%%%%%%%%%
%%%%%%%%%%%%%%%%%%%%%%%%%%%%%%%%%%%%%%%%%%%%%%%%%%%%%%%%%%%%%%%%%%%%%%%%%%%%%%%%%%%%%%%%%%%%%%%%%%%%%%%%%%%%%%%%%%%%%%%%

\begin{document}
\pagestyle{fancy} % Activate own pagestyle

\papertitle{MPG09}{Max-Planck-Gesellschaft. 2009. Regeln zur Sicherung guter
                   wissenschaftlicher Praxis.}

The paper is about rules for good scientific practice, specifically how that is
done at the Max-Planck-Gesellschaft.

First, it gives an introduction about why this paper is relevant, about why we
want to have good scientific practices. It concludes that science in the first
place wants to create new knowledge, not fake it. If we do fake it, we lose
trust from the public and other scientists and ultimately lose the basis of our
work.

There are conditions for good and responsible scientific practice. Some of the
general principles are cooperation, being open to critique and be sceptic about
your own and the results of others. Furthermore, you should have enough
self-control and make your basic assumptions about the research subject clear.
For publication, you should try to make your research contributions freely
available if possible and provide others with a fair review process.
In addition, leadership and supervision should be part of a research facility.
In there, junior scientists should be trained in good scientific research
practices and supervised.
Since the results of research are often times not applied very soon, primary
data and documentation must be kept and made accessible in a readable state for
at least 10 years - anonymized if possible. Publications should make their
results and methods comprehensible and complete.
Then there is a part of issue resolving. A go-to-guy should be assigned.
Whisleblowers and people that have other issues should have a person to talk to
without their names being disclosed.
Same goes for conflict resolution betwee scientific and political, business or
financal interests: science has priority. Therefore one must disclose any
interests or ties beween science and business.

\vspace{8mm}

\papertitle{K+14}{Kramer, Guillory, Hancock. 2014. Experimental Evidence of
                  Massive-scale Emotional Contagion through Social Networks.
                  PNAS. 111, 24 (March 2014), 8788–8790.}

It is known that longer-lasting moods (e.g., depression or happiness) can be
transferred through networks. This paper suggests that this emotional contaigon
also occurs outside of person. The research question is weather \textit{mood can
be transferred outside of in-person interaction through social-networks}.

The hypothesis they provide is if exposure to emotions let people to post
content that was consistent with the exposed emotion. The experiment will
reduce or increase the amount of emotional content in the news feed of persons.
The experiment is split in two: a positive and a negative one by reducing
positive or negative news feed content.
The results were that people who had positive content reduced had a larger
percentage of negative and a smaller percentage of positive words. The opposite
effect also occurred. That means that emotions expressed by friends via online
social networks influence our own moods. However, the effect size is quite small
but may add up and influence other secondary variables like health care costs.

\end{document}

%%%%%%%%%%%%%%%%%%%%%%%%%%%%%%%%%%%%%%%%%%%%%%%%%%%%%%%%%%%%%%%%%%%%%%%%%%%%%%%%%%%%%%%%%%%%%%%%%%%%%%%%%%%%%%%%%%%%%%%%
%%%%%%%%%%%%%%%%%%%%%%%%%%%%%%%%%%%%%%%%%%%%%%%%%%%%%%%%%%%%%%%%%%%%%%%%%%%%%%%%%%%%%%%%%%%%%%%%%%%%%%%%%%%%%%%%%%%%%%%%

% Rules for good scientific practice at Max-Planck
% Why
	% If we dont do it we lose trust from
		% public
		% other scientists
	% We want to create knowledge, not fake it
% Conditions for responsible scientific practice
	% General principles
		% Cooperation (block others etc)
		% Open critique & sceptizism about own and other results
		% Self-control & making assumptions clear
		% Make everything open if possible
		% Fair review process
	% Cooperation
		% Leadership, supervision, conflict resolution and quality control should exist in a research facility
		% Sharing work and integrating it
	% Junior scientists should be trained in scientific research practices
		% Cooperation with universities
		% Supervision of junior scientists / PHDs etc.
	% Primary data & Documentation
		% Must be kept and made accessible in a readable state for at least 10y
	% Data Privacy
		% all data must be anonymized
		% if its not possible, separate the data and the persons information through indirection
	% Publications
		% Results and their methods should be comprehensible and complete
		% Authors should have contributed substantial parts of a publication
	% Go-To-Guy
		% If one is suspicious or about violations of responsible scientific practice
	% Conflict between scientific and political/business/financal interests
		% Science has priority
		% One must disclose any interests or ties beween science and business.
	% Whistleblowers
		% Whitleblowers often have a fear of reprisal if they disclose bad scientific behaviour.
		% Must be protected and name must not disclosed

%%%%%%%%%%%%%%%%%%%%%%%%%%%%%%%%%%%%%%%%%%%%%%%%%%%%%%%%%%%%%%%%%%%%%%%%%%%%%%%%%%%%%%%%%%%%%%%%%%%%%%%%%%%%%%%%%%%%%%%%
%%%%%%%%%%%%%%%%%%%%%%%%%%%%%%%%%%%%%%%%%%%%%%%%%%%%%%%%%%%%%%%%%%%%%%%%%%%%%%%%%%%%%%%%%%%%%%%%%%%%%%%%%%%%%%%%%%%%%%%%

% Mood can be transferred through networks.
% This effect has been criticized
    % Interaction rather than emotion
    % Are nonverbal (silent) cues necessary for contaigon to occur of are verbal ones sufficient?
    % Liking and disliking vs tranferring emotional state.
    % Happiness of others -> Self are depressed
% Research Question
    % Can mood be transferred outside of in-person interaction through social-networks?
% Hypothesis
    % Exposure to emotions let people to post content that was consistent with the exposed emotion.
% Experiment
    % Reducing the amount of emotional content in the News Feed.
    % Positive and negative experiment by reducing positive / negative News Feed content.
        % Control condition: Similar portion of posts were omitted from the News Feed at random.
        % Dependent variables: The percentage of all words produced by a person that was positive or negative.
        % People being exposed to less positive posts should be less positive compared to the control group.
        % People being exposed to less positive posts should be more negative compared to the control group.
% Results
    % People who had positive content reduced had a larger percentage of negative and a smaller percentage of positive words.
    % The opposite effect also occurred.
    % Emotions expressed by friends viao online social networks influence our own moods.
% Discussion
    % News Feed is not directed: Contaigon could not be just the result of interaction with a happy or sad partner
    % Withdrawal effect debunked: Seeing positive posts did not make us negative but positive
    % Small effect sized, but may add up, even in offline behavior

%%%%%%%%%%%%%%%%%%%%%%%%%%%%%%%%%%%%%%%%%%%%%%%%%%%%%%%%%%%%%%%%%%%%%%%%%%%%%%%%%%%%%%%%%%%%%%%%%%%%%%%%%%%%%%%%%%%%%%%%
%%%%%%%%%%%%%%%%%%%%%%%%%%%%%%%%%%%%%%%%%%%%%%%%%%%%%%%%%%%%%%%%%%%%%%%%%%%%%%%%%%%%%%%%%%%%%%%%%%%%%%%%%%%%%%%%%%%%%%%%
